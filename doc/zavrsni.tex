\documentclass[times, utf8, zavrsni]{fer}
\usepackage{booktabs}
\usepackage{glossaries}

\makeglossaries

\newglossaryentry{LSTM}
{
    name=LSTM,
    description={ćelija s dugoročnom memorijom (engl.~\emph{Long short--term memory})}
}

\newglossaryentry{SVM}
{
    name=SVM,
    description={stroj potpornih vektora (engl.~\emph{Support--vector machine})}
}

\newglossaryentry{CNN}
{
	name=CNN,
	description={konvolucijska neuronska mreža (engl.~\emph{Convolutional neural network})}
}

\newglossaryentry{RNN}
{
	name=RNN,
	description={povratna neuronska mreža (engl.~\emph{Reccurent neural network})}
}

\begin{document}

% TODO: Navedite broj rada.
\thesisnumber{000}

% TODO: Navedite naslov rada.
\title{Strojno učenje za analizu sentimenta u mikroblogovima}

% TODO: Navedite vaše ime i prezime.
\author{Ivan Križanić}

\maketitle

% Ispis stranice s napomenom o umetanju izvornika rada. Uklonite naredbu \izvornik ako želite izbaciti tu stranicu.
\izvornik

% Dodavanje zahvale ili prazne stranice. Ako ne želite dodati zahvalu, naredbu ostavite radi prazne stranice.
\zahvala{TODO}

\tableofcontents

\chapter{Uvod}

Mikroblogovi su danas jedan od najčešće korištenih i najčešće proučavanih oblika komunikacije na internetu. Pronalazimo ih na iznimno popularnim društvenim mrežama, kao što su Twitter i Facebook, koji broje milijune korisnika diljem svijeta. Ljudi ih objavljuju u stvarnom vremenu, izražavajući svoje osjećaje, stavove i razmišljanja u svakodnevnom životu. Mnogi događaji i pojave u svijetu dobro su popraćeni reakcijama na drštvenim mrežama, stoga je korisno proučavati velike skupove objava kao izvor stajališta, preferenci, osjećaja i mnogih drugih svojstava koja se daju izvući iz značenja. 

Ovaj se rad konkretno bazira na mikroblogovima društvene platforme Twitter. Takozvani Tweetovi, mikroblogovi platforme Twitter, kratke su poruke sačinjene od najviše 140 znakova. Prvi je objavljen 2005. godine, a dvije godine kasnije dnevno se objavljivalo 5000 mikroblogova. Po zadnjim poznatim podatcima taj broj iznosi preko 500 milijuna objava dnevno.\citep{twitterStats} Radi se o iznimno velikom broju podataka koji kao skup mogu nositi korisne informacije, stoga ne čudi da postoje tvrtke koje u ponudi imaju analizu mikroblogova sa Twittera i drugih društvenih platformi (\textit{brandmentions.com, mention.com}). Povratna informacija korisnika vrijedan je resurs kojim se tvrtke mogu obskrbiti, stoga analiza društvenih platformi ima velik ekonomski i društveni značaj. Obradi tako velikog broja podataka pristupa se tehnikama strojnog učenja, a konkretno područje koje se primjenjuje za ovakve zadatke naziva se obrada prirodnog jezika i još preciznije analiza sentimenta. 

U radu sam se pozabavio problemom klasifikacije mikroblogova na one pozitivnog, neutralnog i negativnog sentimenta. Zadatak odgovara podzadatku A, četvrtog zadatka na natjecanju \emph{ Semeval 2017}, koji je u vrijeme održavanja privukao 39 timova iz cijelog svijeta. Ta je godina bila peta u nizu na kojoj se pojavio isti zadatak, što pokazuje interes zajednice za problem analize sentimenta. U sklopu zadatka napravio sam dva modela za klasifikaciju. Jedan pripada standardnom strojnom učenju i temelji se na \gls{SVM} modelu s linearnom jezgrom, a drugi pripada području dubokog učenja i temelji se na \Gls{LSTM} modelu .
%TODO struktura rada

\chapter{Povezani radovi}


Na temu analize sentimenta napisano je mnogo radova, a velik broj bavi se upravo mikroblogovima s društvenih mreža i to vrlo često upravo Twitterom. Uz to, u sklopu natjecanja \emph{Semeval} neki natjecatelji objavljuju i rad u kojem se osvrću na svoju implementaciju rješenja. Stoga je dostupno puno informacija koje se mogu iskoristiti za vlastitu implementaciju, ali je istovremeno i otežano implementirati neviđeno rješenje. 
%TODO update postotak točnosti
Najbolji rezultat moje implementacije ima točnost od $64\%$, što odgovara 14. mjestu na ljestvici predanih implementacija natjecanja \emph{Semeval 2017}. \citep{semeval2017task4} Prvo mjesto sa točnošću od $68.1\%$ podijelila su dva tima: \textit{DataStories} i \textit{BB\_twtr}. Upravo je tim \textit{BB\_twtr} zaslužan za aktualan \textit{state-of-the-art} model u području analize sentimenta mikroblogova. Njihova trenutna implementacija hvali se da ostvaruje \textit{F1-score} u iznosu od $68.5\%$.

U sljedećih nekoliko odlomaka osvrnut ću se na radove koji su mi služili kao izvor metoda i ostalih informacija koje sam koristio u izradi svoje implementacije.

\section{Rad tima \textit{BB\_twtr} - najuspješniji model današnjice}

Prvi u nizu radova na koje se želim osvrnuti jest rad pobjednika natjecanja \emph{Semeval 2017}, a ujedno i aktulani \textit{state-of-the-art} model u području analize sentimenta mikroblogova. Radi se o radu \textit{BB\_twtr at SemEval-2017 Task 4: Twitter Sentiment Analysis with CNNs and LSTMs} \citep{cliche-2017-bb}. Poblemu su pristupili tehnikama dubokog učenja. Prva faza rada bavi se izradom vektora riječi koji su dalje korišteni u treniranju CNN i LSTM modela mreža. Eksperimentirali su s tri različite tehnike izrade vektora riječi ( \textit{Word2Vec, FastText, GloVe}). U drugoj su fazi nenadziranim učenjem razdijelili sentiment na negativan i pozitivan, jer je prije toga sentiment polariteta u vektorima bio vrlo slab. U trećoj su fazi provodili nazdirano učenje koristeći podatke sa natjecanja i model izgrađen od 10 CNN i 10 LSTM mreža koje koriste različit broj epoha za treniranje i različite vektore riječi. U podzadatku A postigli su točnost od $68.1\%$, a model su koristili i u ostala 4 podzadatka natjecanje te su u svim zadatcima ostvarili najbolji rezultat.

Budući da navode \gls{CNN} i \gls{LSTM} modele mreža kao najbolje u području analize sentimenta, u svojoj sam implementaciju upotrijebio \gls{LSTM} model mreže kako bih se upoznao s njegovim mogućnostima. Umjesto izrade vektora riječi iz velikog skupa mikroblogova, odlučio sam se koristiti gotove vektore iz biblioteke \textit{Spacy} koja koristi vektore izrađene metodom \textit{Word2Vec}. Pri tome gubim prednosti posebnih značajki koje su karakteristične za jezik mikroblogova, a koje bi se mogle pokazati u vektorima nastalim na temelju mikroblogova, ali pristup je jednostavniji i štedi znatnu količinu računalne obrade koja bi bila potrebna za izradu vlastitih vektora.

\section{Rad tima \textit{DataStories}}

U podzadatku A natjecanja \emph{Semeval 2017}, zadatka 4, prvo mjesto dijelila su dva tima, ali tim \textit{DataStories} imao je niži \textit{F1-score}. Svoj su pristup opisali u radu \textit{DataStories at SemEval-2017 Task 4: Deep LSTM with Attention forMessage-level and Topic-based Sentiment Analysi} \citep{datastore-Semeval}. S obzirom na to da su prethodnih godina ostvarili slabije rezultate, dok su timovi koji su koristili pristup dubokog učenja pretežno zauzeli pozicije na vrhu, \textit{DataStories} tim odlučio je skrenuti pažnju s klasičnog strojnog učenja na duboko učenje. Rad su podijelili na dva osnovna koraka: obradu teksta i traniranje modela. Za obradu teksta implementirali su vlastite funkcije koje su primjenjive u općoj upotrebi, ali su usmjerene na obradu mikroblogova s Twittera. Za izradu vektora riječi koristili su 330 milijuna neoznačenih mikroblogova na engleskom jeziku. Na vokabularu od 660 tisuća riječi koristili su \textit{GloVe} metodu izrade vektora. U obradi teksta koristili su vlastiti tokenizator koji je prilagođen Twitteru i posjeduje mogućnost izvlačenja raznih elemenata poput datuma, valuta, emotikona i sličnih sadržaja. Za razliku od njih, u svojoj implementaciji koristim implementaciju tokenizatora iz biblioteke \textit{SpaCy} jer je pristupačna i široko korištena. U daljnoj su obradi primjenili standardne postupke pročišćavanja teksta koji se koriste u obradi prirodnog jezika.

Osvrnuli su se na \gls{CNN} i naglasili problematiku gubitka informacije o poretku riječi prilikom uporabe istih. Iz tog su razloga preferirali \gls{RNN}, konkretnije napredniju izvedenicu koja primjenjuje ćelije s dugoročnom memorijom odnosno \gls{LSTM}.  U svojoj su implementaciji koristili dvoslojni dvosmjerni model s mehanizmom za pozornost koji pospješuje prepoznavanje korisnih težina. U \gls{LSTM} sloju modela koristili su 150 neurona i trenirali s podskupovima od 128 podataka. U testiranju su naveli kako mehanizam pozornosti doprinosi rezultatu za $0.04\%$ te ga stoga nisam implementirao u svoj model.


\section{Rad tima \textit{TakeLab} - pristup klasičnim strojnim učenjem}

Za razliku od velikog broja ekipa na natjecanju, tim \textit{TakeLab} odlučio se za pristup klasičnim metodama strojnog učenja. Koristili su skup ručno izrađenih značajki i trenirali na \gls{SVM} modelu s linearnom jezgrom. Kao značajke koriste \emph{Tf--Idf} i gotove vektore riječi, ali i neke specifične značajke poput leksikona pozitivnih i negativnih riječi te posebnu značajku po kojoj je rad dobio ime: "\textit{Nedavne smrti i moć nostalgije}", odnosno originalni engleski naziv \textit{Recent Deaths and the Power of Nostalgia in Sentiment Analysis in Twitter} \citep{2017-takelab}. Značajka se temelji na činjenici da je sentiment mikroblogova koji spominju nedavno preminute ljude pretežno pozitivan jer se ljudi obično prisjećaju pozitivnih stvari vezanih za pokojnika. Također su iskoristili svojstva nostalgije koja upućuju na pretežno pozitivan sentiment prilikom spominjanja pojmova i pojava iz prijašnjih vremena. Za značajne ljude kreirali su značajke koje opisuju osobe s atributima svojestvenima njohovoj društevnoj ulozi, a za pojmove kojima je često pridjeljenja nekakva ocjena, npr. filmovi, igrice, glazba i slično, napravili su značajke koje donose informacije o uspješnosti i popularnosti pojma. Svojim \gls{SVM} modelom ostvarili su solidan plasman u nekoliko zadataka, a u zadatku kojime se bavi ovaj rad ostvarili su 16. mjesto.

Potaknut njihovim radom, a i mnogim drugima koji koriste ručno izrađene značajke, u svojoj implementaciji iskoristio sam leksikon pozitivnih i negativnih riječi, te sam iskoristio \gls{SVM} model. Naprednije i inovativne značajke koje čine ovaj rad posebnim nisam implementirao.

\chapter{Model}

\section{O zadatku 4 natjecanja \emph{Semeval 2017} i analizi sentimenta u mikroblogovima}

Verzija zadatka s kojim sam se bavio u ovom radu peta je u nizu na najtecanju \emph{Semeval}. Kao i svih prijašnjih godina, zadatak je bio poprilično popularan i privukao 48 timova koji su sudjelovali u različitim podzadatcima. U zadatku se kroz godine pojavilo nekoliko podzadatka kao što su ocjena pripadnosti sentimenta mikrobloga određenoj temi i skaliranje pripadnosti na skali od 1 do 5. Osnovna verzija zadatka bavi se klasifikacijom mikroblogova u tri razine polariteta, preciznije u mirkoblogove pozitivnog, neutralnog i negativnog sentimenta. 

Analiza sentimenta u tekstovima kao što su mikroblogovi s društvenih mreža donosi razne poteškoće s kojima se ne moramo nositi kada je riječ o standardnijim oblicima teksta. Problematične karakteristike mikroblogova su niska ograničenost broja znakova koja uzrokuje sažet izraz, ali i povećava uporabu kolovijalnih izraza, skraćenica i raznih suvremenih novotvorenica koje bismo mogli okarakterizirati kao \emph{slang}. Prisutni su i razni elementi koji ne pripadaju prirodnom jeziku kao što su emotikoni, hiperlinkovi i razne oznake kao npr. oznaka korisničkog imena koja ima oblik \emph{"@user"}. Hiperlinkovi u takvim kratkim tekstovima često nose velik teret značenja, odnosno često tek uz informaciju o sadržaju na koji hiperlink pokazuje možemo pravilno ocijeniti sentiment same poruke. Problem je i u pravopisu, korisnici često mijenjaju riječi radi postizanje vizualnog ili nekog drugog efekta, pa tako možemo naići na tvorevine poput: "ŁoŁ", "ca\$h", "\copyright ool", i slične koje bi bilo poželjno prepoznati i pretvoriti u smislene riječi ili kratice. Također je pristuno nizanje istog slova u riječima poput "\emph{cool}" koje možemo pronaći u obliku kao što je "\emph{cooool}" ili negaciji \emph{"no"} kojoj se često nadodaje zadnje slovo \emph{"o"}. Takvim je riječima također poželjno ukloniti suvišne znakove kako bi se pronašče u rječniku, ali treba imati na umu da takvo ponavljanje znakova nosi značenje u sebi, a koje bismo klasičnim ispravljanjem pravopisa izgubili. Korisno bi bilo prepoznati i skrivene riječi kao što su \emph{"F**k"}. \emph{"S**t"}, \emph{"N***a"} jer su to često riječi koje mogu znatno utjecati na sentiment objave, no to nije tako jednostavan zadatak zbog raznih metoda kojima se takve, često proste riječi, pokušavaju ukomponirati u tekstove.

Kada se odmaknemo od početne obrade teksta nailazimo na nove poteškoće kao što su korištenje sarkazma i učestalost ciničnog tona koji u potpunisti mijenjaju polaritet sentimenta, a koje je vrlo teško prepoznati iz perspektive modela. Ograničenost duljine poruke posljedično donosi manjkavost izraza koji se često bolje razumiju ukoliko se posjeduje znanje o svijetu i vremenu u kojem su napisani, a ne samo o jeziku i značenju istog.

\section{Odabir metoda i pristupa}

U rješavanju problema koristio sam dva različita pristupa kako bih se osvijestio o prednostima i manama jednog i drugog. Prvi pristup pripada klasičnim metodama strojnog učenja i temelji se na vlastaručnoj izradi značajki i upotrebi \gls{SVM} modela. Drugi pristup pripada grani strojnog učenja koja se naziva duboko učenje i  temelji se na značajkama nastalima od vektora riječi  i \gls{LSTM} modelu neuronske mreže.



\subsection{Klasično strojno učenje - \gls{SVM} model}





\chapter{Zaključak}
Zaključak.
\bibliography{literatura}
\bibliographystyle{fer}
\printglossaries

\begin{sazetak}
Sažetak na hrvatskom jeziku.

\kljucnerijeci{strojno učenje, duboko učenje, obrada prirodnog jezika, analiza sentimenta, analiza mikroblogova, Semeval}
\end{sazetak}

% TODO: Navedite naslov na engleskom jeziku.
\engtitle{ 	Machine Learning for Sentiment Analysis in Microblogs}
\begin{abstract}
Abstract.

\keywords{Machine learning, Deep learning, Natural language processing, Sentiment analysis, Microblogs analysis, Semeval.}
\end{abstract}


\end{document}
