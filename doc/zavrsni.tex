\documentclass[times, utf8, zavrsni]{fer}
\usepackage{booktabs}


\begin{document}

% TODO: Navedite broj rada.
\thesisnumber{000}

% TODO: Navedite naslov rada.
\title{Strojno učenje za analizu sentimenta u mikroblogovima}

% TODO: Navedite vaše ime i prezime.
\author{Ivan Križanić}

\maketitle

% Ispis stranice s napomenom o umetanju izvornika rada. Uklonite naredbu \izvornik ako želite izbaciti tu stranicu.
\izvornik

% Dodavanje zahvale ili prazne stranice. Ako ne želite dodati zahvalu, naredbu ostavite radi prazne stranice.
\zahvala{}

\tableofcontents

\chapter{Uvod}
 
Mikroblogovi su danas jedan od najčešće korištenih i najčešće proučavanih oblika komunikacije na internetu. Pronalazimo ih na iznimno popularnim društvenim mrežama, kao što su Twitter i Facebook, koji broje milijune korisnika diljem svijeta. Ljudi ih objavljuju u stvarnom vremenu, izražavajući svoje osjećaje, stavove i razmišljanja u svakodnevnom životu. Mnogi događaji i pojave u svijetu dobro su popraćeni reakcijama na drštvenim mrežama, stoga je korisno proučavati velike skupove objava kao izvor stajališta, preferenci, osjećaja i mnogih drugih svojstava koja se daju izvući iz značenja. 

Ovaj se rad konkretno bazira na mikroblogovima društvene platforme Twitter. Takozvani Tweetovi, mikroblogovi platforme Twitter, kratke su poruke sačinjene od najviše 140 znakova. Prvi je objavljen 2005. godine, a dvije godine kasnije dnevno se objavljivalo 5000 mikroblogova. Po zadnjim poznatim podatcima taj broj iznosi preko 500 milijuna objava dnevno.\citep{twitterStats} Radi se o iznimno velikom broju podataka koji kao skup mogu nositi korisne informacije, stoga ne čudi da postoje tvrtke koje u ponudi imaju analizu mikroblogova sa Twittera i drugih društvenih platformi (\textit{brandmentions.com, mention.com}). Povratna informacija korisnika vrijedan je resurs kojim se tvrtke mogu obskrbiti, stoga analiza društvenih platformi ima velik ekonomski i društveni značaj. Obradi tako velikog broja podataka pristupa se tehnikama strojnog učenja, a konkretno područje koje se primjenjuje za ovakve zadatke naziva se obrada prirodnog jezika i još preciznije analiza sentimenta. 

U radu sam se pozabavio problemom klasifikacije mikroblogova na one pozitivnog, neutralnog i negativnog sentimenta. Zadatak odgovara podzadatku A, četvrtog zadatka na natjecanju Semeval 2017, koji je u vrijeme održavanja privukao 39 timova iz cijelog svijeta. Ta je godina bila peta u nizu na kojoj se pojavio isti zadatak, što pokazuje interes zajednice za problem analize sentimenta. U sklopu zadatka napravio sam dva modela za klasifikaciju. Jedan pripada standardnom strojnom učenju i temelji se na SVM-u sa linearnom jezgrom, a drugi pripada modelima dubokog učenja i temelji se na LSTM modelu.



\chapter{Zaključak}
Zaključak.
\bibliography{literatura}
\bibliographystyle{fer}


\begin{sazetak}
Sažetak na hrvatskom jeziku.

\kljucnerijeci{Ključne riječi, odvojene zarezima.}
\end{sazetak}

% TODO: Navedite naslov na engleskom jeziku.
\engtitle{Title}
\begin{abstract}
Abstract.

\keywords{Keywords.}
\end{abstract}

\end{document}
